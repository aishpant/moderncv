\documentclass[10pt,a4paper,sans,unicode]{moderncv} % Font sizes: 10, 11, or 12; paper sizes: a4paper, letterpaper, a5paper, legalpaper, executivepaper or landscape; font families: sans or roman

\moderncvstyle{classic} % CV theme - options include: 'casual' (default), 'classic', 'oldstyle' and 'banking'
\moderncvcolor{blue} % CV color - options include: 'blue' (default), 'orange', 'green', 'red', 'purple', 'grey' and 'black'
%\renewcommand{\familydefault}{\sfdefault}         % to set the default font; use '\sfdefault' for the default sans serif font, '\rmdefault' for the default roman one, or any tex font name
\nopagenumbers{}                                  % uncomment to suppress automatic page numbering for CVs longer than one page

% Avoid hyphenating words, paragraphs look a lot neater
\usepackage[none]{hyphenat}
\usepackage{soul}

\usepackage{hyperref}

\usepackage[scale= 0.9]{geometry} % Reduce document margins
\setlength{\hintscolumnwidth}{2cm} % Uncomment to change the width of the dates
%column
\setlength{\makecvtitlenamewidth}{10cm} % For the 'classic' style, uncomment to
%adjust the width of the space allocated to your name

% personal data
\name{Aishwarya Pant}{} \social[linkedin]{aishpant}
\social[twitter]{aishpant} \social[github]{aishpant}
\email{aishpant@gmail.com} \homepage{aishpant.dev}


\begin{document}
\makecvtitle
\vspace{-9mm}
%----------------------------------------------------------------------------------------
%	ABOUT
%----------------------------------------------------------------------------------------


\section{About}
\cvitem{}{A strong individual contributor with experience leading the design and implementation of technical features.\newline{}
I advocate for quality, testing and user experience.}


%----------------------------------------------------------------------------------------
%	WORK EXPERIENCE
%----------------------------------------------------------------------------------------

\section{Work Experience}
\cventry{Aug 2018 -- Present}{Backend Engineer}{\textsc{Monzo Bank}}{}{London}{\textnormal{Writing microservices in Go to implement innovative new features and to protect Monzo’s 4 million customers from financial crime.
\vspace{2mm}
\begin{itemize}
    \item \textbf{Designed and implemented decision systems} for automating defunds to blocked customers' account. Defunds are critical piece of infrastructure subject to complex regulation; correctness and robustness is essential. I wrote \textbf{extensive unit tests and acceptance tests} for validating correctness and to make future refactors easier. Nearly 50\% of all fincrime defunds were automated, which saved hundreds of hours per week in operational work. To date, this system has moved over 2 million pounds.
    \vspace{2mm}
    \item Implemented asynchronous \textbf{multi party authorisation} for manual account defunds, simplifying the operational work for our internal staff while ensuring that safety critical operations are secure.
    \vspace{2mm}
    \item Built an \textbf{improved user experience} for customers whose accounts are temporarily blocked as well as those who would be off-boarded from Monzo.
    \vspace{2mm}
    \item Built \textbf{core banking features} like \href{https://monzo.com/blog/2018/11/20/flexible-budgeting}{\ul{financial account summary}} and payee accounts.
    \vspace{2mm}
    \item Owned integration of \href{https://monzo.com/blog/2019/05/08/switch-energy-supplier-through-monzo}{\ul{energy switching}}, contents insurance and a remortgaging service within the Monzo app. This involved \textbf{collaborating with external partners} on designing the product, defining API specifications and writing documentation. Since its launch in 2019, energy switching has generated a revenue of over £500,000.
    \vspace{2mm}
    \item Wrote clear \textbf{technical proposals} for all the projects I worked on, added monitoring and alerts for key services and participated in specialist on-call rotation.
\end{itemize}
}}
\vspace{3mm}

\cventry{Dec 2017 --
Mar 2018}{Outreachy Intern}{\textsc{Linux
Kernel}}{}{Remote}{\textnormal{Wrote a library using python and coccinelle scripts, to improve attribute documentation for the Linux Kernel. This library - abi2doc is available on PyPI and the code is on  \href{https://github.com/aishpant/attribute-documentation}{\ul{GitHub}}.
\vspace{2mm}    
\begin{itemize}
        \item Added documentation for \textbf{312} sysfs attributes, and wrote an extended description of the project on my \href{https://aishpant.github.io/blog/attribute-documentation}{\ul{blog}}.
        \vspace{2mm}
        \item \href{https://git.kernel.org/pub/scm/linux/kernel/git/next/linux-next.git/log/?qt=grep&q=Aishwarya+Pant}{\ul{Link}} to list of patches submitted to the Linux Kernel.
\end{itemize}
}}
\vspace{3mm}

\cventry{July 2015 -- Nov 2017}{Software Development Engineer}{\textsc{Flipkart}}{}{Bengaluru}{\textnormal{Flipkart is one of the largest e-commerce companies in India, with 10 million monthly active users. I primarily worked in the supply chain division of Flipkart.
\vspace{2mm}
\begin{itemize}
	\item Highlights have
been building a user interface in ReactJS for use in transportation
hubs and writing a new sortation service in Java for use in shipment sortation hubs.
\vspace{2mm}
        \item I also worked for 6 months with the Merchandisation
\& Monetisation team, on-boarding new widgets on the landing \& category pages.
\end{itemize}
}}

%----------------------------------------------------------------------------------------
%	EDUCATION
%----------------------------------------------------------------------------------------

\section{Education}
\cventry{2011--2015}{Indraprastha Institute of Information Technology, Delhi}{B.Tech GPA 8.82}{}{\textit{Computer Science}}{}

%----------------------------------------------------------------------------------------
%	SKILLS
%----------------------------------------------------------------------------------------

\section{Skills}
\cvitem{}{\textnormal{Proficient in Go; comfortable with Python, Java, Coccinelle.}}
\cvitem{}{\textnormal{Knowledge of SQL, NSQ, Cassandra, Prometheus, Linux, Git}}

%----------------------------------------------------------------------------------------
%	PUBLICATIONS
%----------------------------------------------------------------------------------------

\section{Publications}
\cventry{2015}{Understanding Thermal Face Detection: Challenges and
Evaluation}{\textsc{Book chapter in "Face Recognition Across the
Electromagnetic Spectrum"}}{}{}{\textnormal{Authors: Janhavi Agrawal, Aishwarya Pant, Tejas Dhamecha, Richa Singh,
Mayank Vatsa}}

%----------------------------------------------------------------------------------------
%	TALKS, AWARDS
%----------------------------------------------------------------------------------------

\section{Talks and Awards}
\cvitem{}{Participated in the \href{https://osseu18.sched.com/event/FxYp/panel-discussion-outreachy-linux-kernel-internship-report-moderated-by-julia-lawall-inria}{\ul{Outreachy panel}} at Open Source Summit EU 2018.}
\cvitem{}{Awarded scholarship to attend KubeCon + CloudNativeCon Europe 2018.}

\end{document}
